\documentclass[12pt]{article}
\oddsidemargin .15in \evensidemargin .0in \topmargin 0in
\columnsep 10pt \columnseprule 0pt
\marginparwidth 19pt \marginparsep 11pt \marginparpush 5pt
\headheight 0pt \headsep 0pt
\textheight 8in
\textwidth 6in 
\pagestyle{plain}
\usepackage{graphicx, latexsym}
\usepackage{alltt}
\usepackage{float}
\usepackage{amsmath}
\newcommand{\widesim}[2][1.5]{
      \mathrel{\overset{#2}{\scalebox{#1}[1]{$\sim$}}}
  }

\newcommand \noi {\noindent}
\newcommand \itx {\indent \indent \indent}
\newcommand \bg {\begin}
\newcommand \en {\end}
\newcommand \mth {\begin{math}}
\newcommand \mthx {\end{math}}
\newcommand \ds {\displaystyle}
\newcommand \mbreak {\\ \vspace{0.1in}}
\begin{document}

\begin{center}    
    \noi \bf 2014 West African Ebola Epidemic: Model Development\\ 
    \vspace{.05in}
    \noi Summer 2014\\
    \vspace{.05in}

    \vspace{.15in}
    \noi Grant Brown\\ 

\end{center}
\section{The SEIR Compartmental Epidemic Model}
\subsection{Introduction}
Compartmental epidemic modeling is a flexible and extensible method of describing epidemic behavior.  
Such techniques rely on the idea that individuals within a population undergoing an epidemic process 
can be categorized by disease state. The most common terms used to describe these disease states are: \\ 

\begin{itemize}
    \item {\bf{Susceptible}}: Individuals capable of contracting the disease of interest. 
    \item {\bf{Exposed}}: Individuals who have contracted the disease, but are not yet infectious. 
    \item {\bf{Infectious}}: Individuals who are capable of spreading the disease.  
    \item {\bf{Recovered/Removed}}: Individuals who have either recovered or been removed from the population. 
\end{itemize}

The stochastic SEIR model has been successfully applied in the past to Ebola by Lekone and Finkenst{\"a}dt (2006), 
making it a good candidate model family for the current outbreak. The development here is broadly simily to the 
aforementioned paper, ``Statistical Inference in a Stochastic Epidemic SEIR Model with Control Intervention: Ebola 
as a Case Study'', with the addition of a simple conditionally autoregressive spatial correlation structure. 

\section{Model Development}

    \subsection{Compartments and Notation}
    {
        \itx
        Denote the spatial locations of interest $\left\{s_i : i = 1, ...,n \right\}$ \\
        Let $d(s_i, s_l) = d_{il}$ be an indicator function which denotes whether two spatial locations share a border.  
        spatial locations. Note that $d(s_i, s_i) = 0$, and that for this data analysis all locations share a border. Should
        this change as the epidemic progresses (by spreading to additional nations), only small changes will be required. \\
         
        Let time (in units appropriate to the data and disease process) be denoted ${t_j : j = 1, ...,T}$ \\
        Let each of these discrete time points be associated with an offset term which describes the relative amount of 
        continuous time over which the discretely indexe data was aggregated. Here, this offset is the time between WHO reports. 
        Define the following components for each $s_i$ and $t_j$: \mbreak

        All components indexed by time and space in libspatialSEIR are layed out as arrays of double precision elements
        in column major order with T rows and n columns. 
    }
        \begin{itemize}
            \item {${N_{ij}}$} is the population size.
            \item {${S_{ij}}$} is the count of susceptible individuals.
            \item {${E_{ij}}$} is the count of exposed individuals.
            \item {${I_{ij}}$} is the count of infectious individuals. 
            \item {${R_{ij}}$} is the count of recovered/removed individuals.
            \item {${E^*_{ij}}$} is the number of newly exposed individuals.
            \item {${I^*_{ij}}$} is the number of newly infectious individuals. We assume that the observed data gives an accurate count of this value. 
            \item {${R^*_{ij}}$} is the number of newly recovered/removed individuals.
        \end{itemize}
        
        Let {$\bf{N_j} = \bf{S_j + E_j + I_j + R_j}$} for all $j$ rows of the disease process compartments. As population size is not 
        expected to change dramatically over the time scale of this epidemic, $N$ is held fixed for each spatial location. 
        In addition let ${\bf{S_0, E_0, I_0,}}$ and ${\bf{R_0}}$ denote the $n$-vectors of unknown compartment sizes at the 
        start of the modeling period. Here, $I_0$ is initialized at $86$ in Guinea and $0$ elsewhere, as this is the first available
        estimate of the number of infectious individuals. The rest of the population was considered susceptible, as is usual for these models.  
        

    \vspace{0.15in}
    \subsection{Disease Evolution Process Model}
    Given the values of the aforementioned parameters, the disease process evolves forward in time 
    as one would expect based on the definitions.  
    \begin{center}


        $\bf{S_{j+1}} = \bf{S_j} - \bf{E_j^*}$\mbreak
        $\bf{E_{j+1}} = \bf{E_j} - \bf{I_j^*} + \bf{E_j^*}$\mbreak
        $\bf{I_{j+1}} = \bf{I_j} - \bf{R_j^*} + \bf{I_j^*}$\mbreak
        $\bf{R_{j+1}} = \bf{R_j} + \bf{R_j^*}$\mbreak
    \end{center}
    \vspace{0.15in}

    While models of this form are often fit using deterministic systems of ordinary or partial differential equations, 
    libspatialSEIR uses a heirarchical Bayesian framework in order to adequately capture the inherent variability in the model parameters.
    To complete the temporal process model, specify the following chain binomial relationship: 
    \vspace{0.15in}
    {


        $\{E_{ij}^* | \pi^{(SE)}_{ij}, S_{ij} \} \widesim{ind}\ binom(S_{ij}, \pi^{SE}_{ij})$ \mbreak

        $\{I_{ij}^* | \pi^{(EI)}, E_{ij} \} \widesim{ind}\ binom(E_{ij}, \pi^{(EI)})$\mbreak

        $\{R_{ij}^* | \pi^{(IR)}, I_{ij}\} \widesim{ind}\ binom(I_{ij}, \pi^{(IR)})$\mbreak
    }
\subsection{Transition Probability Model}

        While $\pi^{EI}$ and $\pi^{IR}$ can be easily parameterized with the usual exponential time assumption, more care must be given to the development 
        of a model for the $\left\{\pi^{SE}_{ij} \right\}$. This set of parameters 
        describes the actual infection process and must account for predictor variables as well as the 
        spatial structure of $\left\{ s_i \right\}$. 

\subsection{Infection Process - CAR Model Motivation}

Define $\delta_{ij}$ to be the proportion of persons who are infectious in spatial unit $s_i$ 
at time $t_j$. Then, letting $\rho f(d_{il})$ denote the scaled standardized neighborhood matrix component for spatial 
locations $s_i$ and $s_l$, we can derive (with minimal further assumptions):

\begin{center}
    $\displaystyle \pi^{(SE)}_{ij} = 1- exp\left\{-\delta_{ij}e^{\theta_{i}} - \rho \sum_{\left\{ l \ne i  \right\}}
        (f(d_{il})\delta_{il}e^{\theta_{l}})   \right\}$ 
\end{center}
 
\section{Basic Reproductive Number}
The basic reproductive number, $\mathcal{R}_0$, is an important quantity in epidemiology. While the interpretation must be adapted to 
the problem of interest, in general terms the basic reproductive number captures the expected number of secondary 
infections produced by a single infected individual in an entirely susceptible population. 

Using the next generation matrix approach to $\mathcal{R}_0$ calculation, we first define the matrix $G$ such 
that $G_{i,l}(t_j)$ is the expected number of infections in spatial location $s_i$ caused by a single infected
individual in location $s_l$ at time $t_j$.

With this matrix constructed, the basic reproductive number can be immediately calculated as the dominant eigenvalue. 


\section{Final Notes}

The sampling scheme uses variants of Metropolis sampling (not Metropolis-Hastings, as symmetric proposal distributions are employed) 
are used, in combination with automatic tuning capabilities which enable libspatialSEIR to choose tuning parameters with reasonable acceptance rates. 

\end{document}
