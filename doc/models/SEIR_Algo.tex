\documentclass[12pt]{article}
\oddsidemargin .15in \evensidemargin .0in \topmargin 0in
\columnsep 10pt \columnseprule 0pt
\marginparwidth 19pt \marginparsep 11pt \marginparpush 5pt
\headheight 0pt \headsep 0pt
\textheight 8in
\textwidth 6in 
\pagestyle{plain}
\usepackage{graphicx, latexsym}
\usepackage{alltt}
\usepackage{float}
\usepackage{amsmath}

\newcommand \noi {\noindent}
\newcommand \itx {\indent \indent \indent}
\newcommand \bg {\begin}
\newcommand \en {\end}
\newcommand \mth {\begin{math}}
\newcommand \mthx {\end{math}}
\newcommand \ds {\displaystyle}
\newcommand \mbreak {\\ \vspace{0.1in}}
\begin{document}

\begin{center}    
    \noi \bf libspatialSEIR: Model, Algorithm, and Implementation\\
    \vspace{.05in}
    \noi Summer 2014\\
    \vspace{.05in}

    \vspace{.15in}
    \noi Grant Brown\\ 

\end{center}
\section{The SEIR Compartmental Epidemic Model}
\subsection{Introduction}
Compartmental epidemic modeling is a flexible and extensible method of describing epidemic behavior.  
Such techniques rely on the idea that individuals within a population undergoing an epidemic process 
can be categorized by disease state. The most common terms used to describe these disease states are: \\ 

\begin{itemize}
    \item {\bf{Susceptible}}: Individuals capable of contracting the disease of interest. 
    \item {\bf{Exposed}}: Individuals who have contracted the disease, but are not yet infectious. 
    \item {\bf{Infectious}}: Individuals who are capable of spreading the disease.  
    \item {\bf{Recovered/Removed}}: Individuals who have either recovered or been removed from the population. 
\end{itemize}

Persons in a population are assumed to move through these disease categories sequentially according to the 
disease process, though in practice these disease states are combined in different ways. For example, a simple epidemic model
might use an S-I-R structure in which individuals become immediately infectious (ie, there is no latent period), and 
are subsequently removed from the population (ie, assuming permanent immunity). For a disease which conferrs only 
temporary immunity, such as influenza, many
researchers have employed S-I-R-S models, which include a potential for previously recovered individuals to be reintroduced
to the susceptible population. Similarly, SEIR and SEIRS models introduce a latent period during which exposed individuals
are not yet, but will become, infectious. In cases where the reintroduction process is either complicated, or data is sparse, 
researchers can employ a "Serial SEIR" model, which simply re-sets the susceptible population at regular intervals.\\ 

This diverse family temporal structures has been generalized to allow epidemics to be modeled over space, as well as time.  

\subsection{TL;DR - What is libsaptialSEIR for?}

libspatialSEIR was designed to provide a computationally efficient and user friendly method to fit several important epidemic 
models in the Bayesian spatial SEIRS family. In particular, spatial and single location models are allowed to evolve through 
time via one of several SEIRS-like structures, with a particular effort to allow the inclusion of linear predictors to drive
important model parameters. Notable examples include:

\begin{itemize}
    \item Single location SEIR and SEIRS models.   
    \item Spatial SEIR and SEIRS models, employing a user specified measure of distance between spatial locations. 
    \item Serial SEIR models, both for single locations and spatially indexed data. 
\end{itemize}

The focus on linear predictors provides an additional layer of flexibility to these models, as they can accommodate 
temporal basis functions for epidemic shape fitting, intervention indicators for public health efforts, demographic 
effects on population mixing, and anything else that can be expressed as a linear combination of unknown, normally
distributed parameters.\\ 


Example code for each of these models is included in the \textit{scripts} directory, and several full tutorials are available in the 
tutorial subdirectory of \textit{doc} folder. 

\subsection{SEIRS Model Family - Formal Development}
        Developed below is the most general member of the SEIRS family which can be fit with libspatialSEIR, 
        namely the spatial SEIRS model. Important special cases are discussed in the following section.

    \subsubsection{Compartments and Notation}
    {
        \itx
        Denote the spatial locations of interest $\left\{s_i : i = 1, ...,n \right\}$ \\
        Let $d(s_i, s_l) = d_{il}$ define a measure of distance between 
        spatial locations. Note that $s_i$ and $s_l$. $d(s_i, s_i) = 0$, and $d(s_i, s_l) = d(s_l, s_i)$ \\
        
        
        Let time (in units appropriate to the data and disease process) be denoted ${t_j : j = 1, ...,T}$ \\
        For now, assume that all time points are equally spaced - the library allows the inclusion of an offset term 
        to relax this assumption, but for illustrative purposes we'll keep the notation simple here. 
        Define the following components for each $s_i$ and $t_j$: \mbreak
    }
        \begin{itemize}
            \item {$y_{ij}$} is the observed data.
            \item {$\bf{N_{ij}}$} is the population size
            \item {$\bf{S_{ij}}$} is the count of susceptible individuals
            \item {$\bf{E_{ij}}$} is the count of exposed individuals
            \item {$\bf{I_{ij}}$} is the count of infectious individuals
            \item {$\bf{R_{ij}}$} is the count of recovered/removed individuals
            \item {$\bf{S^*_{ij}}$} is the number of newly susceptible individuals
            \item {$\bf{E^*_{ij}}$} is the number of newly exposed individuals
            \item {$\bf{I^*_{ij}}$} is the number of newly infectious individuals
            \item {$\bf{R^*_{ij}}$} is the number of newly recovered/removed individuals
        \end{itemize}
        Let {$\bf{N_j} = \bf{S_j + E_j + I_j + R_j}$} for all $j$\\
        In addition let ${\bf{S_0, E_0, I_0,}}$ and ${\bf{R_0}}$ denote the $n$-vectors of unknown compartment sizes at the 
        start of the modeling period. 
        

    \subsection{Data Model}
    An obvious first step is to specify the data model. 

    \vspace{0.15in}

    \begin{center}
        $ \{y_{ij}\ | I^*_{ij}\} \sim\ (indep.)\ g(I^*_{ij}, \Theta)  $
    \end{center}

    Currently, libspatialSEIR only supports the identity data model, in which: 
    \begin{center}
        $g(I^*_{ij}) = I^*_{ij}$  
    \end{center}
    with probability one. This is under active development. \\

    \vspace{0.15in}

    \subsubsection{Disease Evolution Process Model}
    Given the values of the aforementioned parameters, the disease process evolves forward in time 
    as one would expect based on the definitions.  
    \begin{center}
        $\bf{S_{j+1}} = \bf{S_j} - \bf{E_j^*} + \bf{S_j^*}$\mbreak
        $\bf{E_{j+1}} = \bf{E_j} - \bf{I_j^*} + \bf{E_j^*}$\mbreak
        $\bf{I_{j+1}} = \bf{I_j} - \bf{R_j^*} + \bf{I_j^*}$\mbreak
        $\bf{R_{j+1}} = \bf{R_j} - \bf{S_j^*} + \bf{R_j^*}$\mbreak
    \end{center}
    \vspace{0.15in}

    While models of this form are often fit using deterministic systems of differential equations, libspatialSEIR uses
    a heirarchical Bayesian framework in order to adequately capture the inherent variability in the model parameters.
    To complete the temporal process model, specify the following chain binomial relationship: 
    \vspace{0.15in}
    {

        $\{S_{ij}^* | \pi_j^{(RS)}, R_{ij}\} \sim\ (indep.)\  binom(R_{ij}, \pi_j^{(RS)})$\mbreak

        $\{E_{ij}^* | \pi^{(SE)}_{ij}, S_{ij} \} \sim (indep.)\ binom(S_{ij}, \pi^{SE}_{ij})$ \mbreak

        $\{I_{ij}^* | \pi^{(EI)}, E_{ij} \} \sim\ (indep.)\ binom(E_{ij}, \pi^{(EI)})$\mbreak

        $\{R_{ij}^* | \pi^{(IR)}, I_{ij}\} \sim\ (indep.)\ binom(I_{ij}, \pi^{(IR)})$\mbreak
    }
\subsection{Transition Probability Model}

        While $\pi^{EI}$, and $\pi^{IR}$ can be easily parameterized, more care must be given to the development 
        of a model for the $\left\{\pi^{SE}_{ij} \right\}$ and $\left\{\pi^{RS}_j \right\}$. The first of these 
        describes the actual infection process and must account for predictor variables as well as the 
        spatial structure of $\left\{ s_i \right\}$. The second drives the reinfection process, which captures the
        effects of diminishing temporary immunity, disease strain mutation, and the introduction of new infectious 
        agents. These important components are examined in the following sections. 

\subsection{Infection Process - CAR Model Motivation}

Consider the process by which people become infected with a communicable disease. 
Namely, consider the situation in which a person `A' has contacted another person, `B', 
who is infectious (for some suitable definition of contacted). 
Let $p$ be the probability that person `A' becomes infected with the disease, and
let $q=1-p$. Now we introduce a number of assumptions:

\begin{itemize}

    \item Assume that the number of `contacts' $K_i$ between a person of interest 
    and other individuals within a spatial unit $s_i$ at a given time point follows a poisson 
    distribution:\\
    \begin{center}
        $K_j \sim Po(\lambda_i)$
    \end{center}
    \item Assume that when individuals travel to other spatial locations, their 
        contact behavior is well modeled by the contact behavior of that spatial unit (when in Rome).  
    \item Contact between spatial locations is proportional to some known function $f(d_{il})$
        of the chosen distance metric between the centroids of $s_i$ and $s_l$
\end{itemize}

Define $\delta_{ij}$ to be the proportion of persons who are infectious in spatial unit $s_i$ 
at time $t_j$. Then, letting $Inf(s_i, t_j)$ denote the event that a person becomes 
infected from contact within spatial unit $s_i$ at time $t_j$, we can derive:

\begin{center}

    $P(Inf(.,t_j)) = 1 - P(!Inf(s_i, t_j)) \cdot  P(!Inf(s_{-i}, t_j))$ \mbreak
    where \mbreak 
    $P(!Inf(s_i, t_j)) = E(!Inf(s_i, t_j)) = E(E(!Inf(s_i, t_j)|K_i=k_i))$  \mbreak 
    $\displaystyle =E(((1-\delta_{ij})q)^{k_i})$  \mbreak 
    $\displaystyle = \sum_{k=0}^{\infty} ((1-\delta_{ij})q)^k(\frac{\lambda_i^ke^{-\lambda_i}}{k!})$\mbreak
    $\displaystyle =  \sum_{k=0}^{\infty} q_{ij}^k (\frac{\lambda_i^ke^{-\lambda_i}}{k!})$\mbreak
    $\displaystyle = \frac{e^{-\lambda_i}}{e^{-q_{ij}\lambda_i}}(1)$
    $ = e^{-\lambda_i\cdot(1-q_{ij})} $
    $ = e^{-\lambda_i \cdot p_{ij}} $
    $ = e^{-\lambda_i \cdot (\delta_{ij}p)}$ \mbreak
    Therefore,  $P(Inf(s_i, t_j)) = 1 - e^{-\lambda_i \cdot (\delta_{ij}p)} $ \mbreak
    Similarly,\mbreak
    $\displaystyle P(!Inf(s_{-i}, t_j)) = \prod_{\left\{l \ne i\right\}}\left[P(!Inf(s_l, t_j)) \right]$ \mbreak
    $\displaystyle = \prod_{\left\{ l \ne i \right\}}\left[E(!Inf(s_{-i}, t_j))\right] $
    $\displaystyle = \prod_{\left\{ l \ne i \right\}}\left[E(E(!Inf(s_{-i}, t_j)|K_i=k_i)) \right]$  \mbreak 
    $\displaystyle = \prod_{\left\{ l \ne i \right\}}\left[ E((1-\delta_{lj})q)^k )  \right]$\mbreak 
    $\displaystyle = \prod_{\left\{ l \ne i \right\}}\left[ \sum_{k=0}^{\infty}(q_{lj}(i))^k\frac{(\lambda_l\cdot f(d_{il}))^ke^{-\lambda_l \cdot f(d_{il})}}{k!}    \right] $
    $\displaystyle = \prod_{\left\{ l \ne i \right\}}\left[ \frac{e^{-\lambda_l \cdot f(d_{il})}}{e^{-q_{lj}\lambda_l f(d_{il})}}(1)  \right]$\mbreak
    $\displaystyle = \prod_{\left\{ l \ne i \right\}}\left[ e^{-\lambda_l \cdot f(d_{il}) p_{lj}}  \right]$
    $\displaystyle = \prod_{\left\{ l \ne i \right\}}\left[ e^{-\lambda_l\cdot f(d_{il}) \cdot (\delta_{lj}p) }  \right]$\mbreak
    $\displaystyle = exp\left\{\sum_{\left\{ l \ne i \right\}}\left[p\lambda_l\delta_{jl}f(d_{il})   \right] \right\}$ \mbreak
    Thus, for the probabilty of infection for a person living in $s_i$ at time $t_j$ we have: \mbreak 
    $\displaystyle 1-\left(e^{-\lambda_i \cdot (\delta_{ij}p)}\right) \left( 
        e^{\left\{\sum_{\left\{ l \ne i \right\}}\left[p\lambda_l\delta_{jl}f(d_{il})   \right] \right\}}\right)
     $ \mbreak 

    $\displaystyle = 1- exp\left\{-\delta_{ij}e^{\theta_{i}} - \sum_{\left\{ l \ne i  \right\}}
        (f(d_{il})\delta_{il}e^{\theta_{l}})   \right\}$ 
    , where $\theta_{v} = log(\lambda_{v}p)$
 
\end{center}

Currently, libspatialSEIR supports distance functions of the form:

\begin{center}
    $f(d_{il}) = \rho \cdot (d_{il})^{-\frac{1}{2}}$
\end{center}

While not required, the option to re-scale this distance matrix to be row stochastic is available to ensure a 
proper posterior distribution. 

Support for other distance functions, such as the gravity model, is planned. \mbreak


A few miscellaneous notes: 
\begin{enumerate}
    \item By defining $f(d_{ii})$ to be equal to $1$ for all $i$, the above expression has a simple 
        matrix form. 
    \item WLOG, we can make the mixing parameters dependent on space and time, defining $\lambda_{v_1, v_2}$,
        and correspondingly $\theta_{v_1,v_2}$
    \item When constructing full conditional distributions, it is important to keep in mind the constraint:
        $S+E+I+R=N$, as well as the fact that $p_{i,j}^{(SE)}$ does, in fact, depend on the value of $I$.
        This is easy to neglect and hard to debug.  \\
        
\end{enumerate}

This spatial probability structure belongs to the CAR, or Conditionally Auto-Regressive, class
of spatial dependence structures, and can accommodate several different data types. For example, 
in the case where spatial data is indexed by discrete areal units, a neighborhood matrix may define
the requisite `distance' between spatial locations. On the opposite end of the spectrum, spatial locations 
may occur on a continuum (ie, latitude and longitude). In this case, a more usual distance definition of 
distance can be employed (with or without an explicit spatial range). libspatialSEIR is optomized for the 
second case (ie, sparse matrix methods are not employed), though is perfectly
capable of fitting sparser spatial (or entirely non-spatial) models.\\ 


\subsection{Re-infection Process}


To model the $\pi_j^{RS}$, some structure of lower than $T$ dimensions is desireable to reduce potential 
identifiability issues. A covariate structure constructed to capture natural variation in this quantity,
a set of trigonometric basis functions with appropriate period, for example, will do nicely. Let $X(\pi_j^{RS})$ and $\beta_{\pi^{RS}}$ 
denote the covariate vector and corresponding regression parameter estimates 
for the j'th R to S transition probability respectively. Any number of link functions might be appropriate here, 
however due the the ease with which it can be generalized to inconsistent temporal indices the following (generalized from
Lekone and Finkenst{\"a}dt (2006)) 
is used:

\begin{center}
    $\pi_j^{RS} = 1-e^{-exp(X(\pi_j^{RS}) \beta_{\pi^{RS}})}$
\end{center}
 
\section{Basic Reproductive Number}

The basic reproductive number, $\mathcal{R}_0$, is an important quantity in epidemiology. While the interepretation must be adapted to 
the problem of interest, in general terms the basic reproductive number captures the expected number of secondary 
infections produced by a single infected individual in an entirely susceptible population. 

Using the next generation matrix approach to $\mathcal{R}_0$ calculation, we first define the matrix $G$ such 
that $G_{i,l}(t_j)$ is the expected number of infections in spatial location $s_i$ caused by a single infected
individual in location $s_l$ at time $t_j$.

Defining the relevant infection event for a person indexed by $k$ to be: $I_k(s_i, s_l, t_j)$, we see that the expected number of such
infections is:

\begin{center}
    $\displaystyle E\big[ \sum_{k=0}^{N_{i,j}}(I_k(s_i, s_l, t_j)) \big]$ \mbreak
    $\displaystyle = \sum_{k=0}^{N_{i,j}}\cdot P(I_k(s_i, s_l, t_j)) = N_{i,j}\cdot P(I_k(s_i, s_l, t_j)) $ \mbreak 
    Where, as before: \mbreak
    $ P(I_k(s_i, s_l, t_j)) = 1-exp\left\{-f(d_{il})\delta_{lj}e^{\theta_{l}} \right\}$ \mbreak
    This gives: \mbreak
    $G_{i,l}(t_j) = \frac{N_{i,j}}{I_{l,j}}\cdot \left[1-exp\left\{-f(d_{il})\delta_{lj}e^{\theta_{l}} \right\}\right]$ \mbreak
    Additionally recall the diagonal case, where $d_{ii} = 0$ and $f(0) = 1$:\mbreak 
    $G_{i,i}(t_j) = \frac{N_{i,j}}{I_{i,j}}\cdot \left[1-exp\left\{-\delta_{ij}e^{\theta_{i}} \right\}\right]$ \mbreak
    $ = \delta_{ij}^{-1}\cdot \left[1-exp\left\{-\delta_{ij}e^{\theta_{i}} \right\}\right]$ \mbreak
\end{center}

With this matrix constructed, the basic reproductive number can be immediately calculated as the dominant eigenvalue. 


\section{Posterior Distribution and Full Conditionals}
Bringing together the aforementioned spatio-temporal structures we can define the requisit prior distributions and 
deterministic relationships among parameters, and thus construct the requisit posterior distribution. 

\subsection{Summary of Distribution Components}
\noi $\{y_{ij} | I^*_{ij}\} \sim\ (indep.)\ g(I^*_{ij}, \Theta)$\\

\noi $\{S_{ij}^* | \pi_j^{(RS)}, R_{ij}\} \sim\ (indep.)\  binom(R_{ij}, \pi_j^{(RS)})$\\

\noi $\{E_{ij}^* | \pi^{(SE)}_{ij}, S_{ij} \} \sim (indep.)\ binom(S_{ij}, \pi^{SE}_{ij})$ \\

\noi $\{I_{ij}^* | \pi^{(EI)}, E_{ij} \} \sim\ (indep.)\ binom(E_{ij}, \pi^{(EI)})$\\

\noi $\{R_{ij}^* | \pi^{(IR)}, I_{ij}\} \sim\ (indep.)\ binom(I_{ij}, \pi^{(IR)})$\\

\noi $\gamma^{(IR)} \sim\ gamma(\alpha^{(IR)}, \beta^{(IR)})$\\ 

\noi $\gamma^{(EI)} \sim\ gamma(\alpha^{(EI)}, \beta^{(EI)})$\\ 

\noi $\left\{ \theta_{ij}\right\} \sim \mathcal{N}(\eta_{ij}, \sigma^2_{\theta})$ \\

\noi $\left\{ \beta \right\} \sim \mathcal{N}(0, \tau^2_\beta) $\\ 

\noi $\left\{ \beta_{\pi^{RS}} \right\} \sim \mathcal{N}(0, \tau^2_{RS}) $\\ 

\noi $\sigma^2_{\theta} \sim \Gamma(\alpha_\theta, \beta_\theta)$\\

\noi $\rho \sim U(0,1)$


\subsection{Deterministic Functions}

\noi $S = f_S(S_0, E^*_0, S^*_0, S^*, E^*)$ \\

\noi $E = f_E(E_0, I^*_0, E^*_0, E^*, I^*)$ \\

\noi $I = f_I(I_0, R^*_0, I^*_0, I^*, R^*)$ \\

\noi $R = f_R(R_0, S^*_0, R^*_0, R^*, S^*)$ \\

\noi $\displaystyle log(\pi^{(SE)}_{ij}) = -\delta_{ij}e^{\theta_{ij}} - \sum_{\left\{ l \ne i \right\}}d_{il}\delta_{il}e^{\theta_{il}}$\\

\noi $\left\{\pi_{EI}  \right\} = 1-exp{-\gamma^{(EI)}}$\\

\noi $\left\{\pi_{IR}  \right\} = 1-exp{-\gamma^{(IR)}}$\\

\noi $log(\pi_j^{RS}) = -X(\pi_j^{RS}) \beta_{\pi^{RS}}$\\

\noi $d_{il} = f(\rho, s_i, s_l)$\\

\noi $\delta_{ij} = \frac{I_{ij}}{N_{ij}}$ \\

\noi $\eta_{ij} = X_{ij}\beta$\mbreak

\subsection{Posterior Distribution}

\begin{center}
\begin{multline}
\displaystyle
log(p(\theta,\beta,\rho,S^*,E^*,R^*|.)) \propto \Bigg[ 
    \sum_{i=1}^n \bigg\{ \sum_{j=1}^T
        \Big\{ ln(g(y_{ij}|I^*_{ij})) + \\ 
            (S^*_{ij}log(\pi_j^{(RS)}) + (R_{ij} - S^*_{ij})log(1-\pi_j^{(RS)})) \\
            + (E^*_{ij}log(\pi_{ij}^{(SE)})) + (S_{ij} - E^*_{ij})log(1-\pi_{ij}^{(SE)}) \\
            + (I^*_{ij}log(\pi^{(EI)})) + (E_{ij} - I^*_{ij})log(1-\pi^{(EI)}) \\
            + (R^*_{ij}log(\pi^{(IR)})) + (I_{ij} - R^*_{ij})log(1-\pi^{(IR)}) \\
    - \frac{1}{2}log(\sigma^2_{\theta}) - \frac{(\theta_{ij}-\eta_{ij})^2}{2\sigma^2_{\theta}}\Big\}\bigg\} \\
    + \sum_{k = 1}^K\bigg\{-\frac{\beta^2_k}{10}\bigg\}
            + (log(\pi(\sigma^2_{\theta})))
            + (log(\pi(\rho)))\\
            + \bigg\{ -\frac{\tau^2_{RS}}{2}\|\beta_{\pi_j^{RS}}\|  \bigg\} \\ 
        + log(\pi(\gamma^{(EI)})) + log(\pi(\gamma^{(IR)}))\Bigg]\\
\end{multline}
\end{center}

For simplicity, and because the model is sufficiently flexible without the added over-dispersion, set $\theta_{ij} = \eta_{ij}$ with probability 1. This 
gives the following simplified posterior distribution:

\begin{center}
\begin{multline}
\displaystyle
log(p(\theta,\beta,\rho,S^*,E^*,R^*|.)) \propto \\
\Bigg[ \sum_{i=1}^n \bigg\{ \sum_{j=1}^T
        \Big\{
            ln(g(y_{ij}|I^*_{ij})) \\ 
            + log\bigg(\dbinom{R_{ij}}{S^*_{ij}}\bigg) + (S^*_{ij}log(\pi_j^{(RS)}) + (R_{ij} - S^*_{ij})log(1-\pi_j^{(RS)})) \\
            + log\bigg(\dbinom{S_{ij}}{E^*_{ij}}\bigg) + (E^*_{ij}log(\pi_{ij}^{(SE)})) + (S_{ij} - E^*_{ij})log(1-\pi_{ij}^{(SE)}) \\
            + log\bigg(\dbinom{E_{ij}}{I^*_{ij}}\bigg) + (I^*_{ij}log(\pi^{(EI)})) + (E_{ij} - I^*_{ij})log(1-\pi^{(EI)}) \\
            + (log\bigg(\dbinom{I_{ij}}{R^*_{ij}}\bigg) + R^*_{ij}log(\pi^{(IR)})) + (I_{ij} - R^*_{ij})log(1-\pi^{(IR)}) \Big\}\bigg\} \\
    + \sum_{k = 1}^K\bigg\{-\frac{\beta^2_k}{10}\bigg\}
            + (log(\pi(\rho)))\\
            +  \bigg\{ -\frac{\tau^2_{RS}}{2}\|\beta_{\pi_j^{RS}}\|  \bigg\} \\ 
            + \bigg\{ -\frac{\tau^2_{RS}}{2}\|\beta_{\pi_j^{RS}}\|  \bigg\} \\ 
            + log(\pi(\gamma^{(EI)})) + log(\pi(\gamma^{(IR)}))\Bigg]\\
\end{multline}
\end{center}

\subsection{Full Conditional Distributions}

\begin{itemize}
    \item{Full Conditional For $S^*$}
%\subsubsection{$p(S^*|.)$}
    \begin{center}
        \begin{multline}
        \displaystyle
        log(p(S^*|.))\propto 
            \sum_{j=1}^T\sum_{i=1}^n\Big\{log\bigg(\dbinom{f_R(R^*, S^*, A_0)_{ij}}{S^*_{ij}}\bigg) \\ + log(\pi_j^{(RS)})\{S^*_{ij}\} + 
                log(1-\pi_j^{(RS)})\{f_R(R^*, S^*, A_0)_{ij} - S^*_{ij}\}\\ 
            + log\bigg(\dbinom{f_S(S^*, E^*, A_0)_{ij}}{E^*_{ij}}\bigg) + f_S(S^*, E^*, A_0)_{ij}log(1-\pi_{ij}^{(SE)})\Big\}
        \end{multline}
    \end{center}

    \item{Full Conditional for $E^*$}
%\subsubsection{$p(E^*|.)$}
    \begin{center}
    \begin{multline}
        \displaystyle
        log(p(E^*|.))\propto \sum_{i=1}^n \sum_{j=1}^T\bigg\{log\bigg(\dbinom{f_S(S^*, E^*, A_0)_{ij}}{E^*_{ij}}\bigg) \\ 
                + (E^*_{ij}log(\pi_{ij}^{(SE)})) + (f_S(S^*, E^*, A_0)_{ij} - E^*_{ij})log(1-\pi_{ij}^{(SE)})\bigg\} \\
                + log\bigg(\dbinom{f_E(E^*, I^*, A_0)_{ij}}{I^*_{ij}}\bigg) + log(1-\pi^{(EI)})\sum_{i=1}^n\sum_{j=1}^T\{f_E(E^*, I^*, A_0)_{ij}\}\\
    \end{multline}
    \end{center}

    \item{Full Conditional for $I^*$}
%\subsubsection{$p(E^*|.)$}
    \begin{center}
    \begin{multline}
        \displaystyle
        log(p(I^*|.))\propto \sum_{i=1}^n \sum_{j=1}^T\bigg\{log(g(y_{ij}|I^*_{ij})) + log\bigg(\dbinom{f_E(E^*, I^*, A_0)_{ij}}{I^*_{ij}}\bigg)\\
                + (I^*_{ij}log(\pi^{(EI)})) + (f_E(E^*, I^*, A_0)_{ij} - I^*_{ij})log(1-\pi^{(EI)})\bigg\} \\
                + log\bigg(\dbinom{f_I(I^*, R^*, A_0)_{ij}}{R^*_{ij}}\bigg) + log(1-\pi^{(IR)})\sum_{i=1}^n\sum_{j=1}^T\{f_I(I^*, R^*, A_0)_{ij}\}\\
    \end{multline}
    \end{center}



    \item{Full Conditional for $R^*$}
%\subsubsection{$p(R^*|.)$}
    \begin{center}
    \begin{multline}
        \displaystyle
        log(p(R^*|.)) \propto \sum_{i=1}^n\sum_{j=1}^T\bigg\{log\bigg(\dbinom{f_I(I^*, R^*, A_0)_{ij}}{R^*_{ij}}\bigg)\\ + log(\pi^{(IR)})\{R^*_{ij}\}
            + log(1-\pi^{IR})\{f_I(I^*, R^*, A_0)_{ij} - R^*_{ij}\}\\
            + log\bigg(\dbinom{f_R(R^*, S^*, A_0)_{ij}}{S^*_{ij}}\bigg) + log(1-\pi_j^{(RS)})\{f_R(R^*, S^*, A_0)_{ij}\}\\ 
            + log(\pi_{ij}^{(SE)})\{E^*_{ij}\} + log(1-\pi_{ij}^{SE})\{f_S(S^*, E^*, A_0)_{ij} - E^*_{ij}\}\bigg\}\\ 
    \end{multline}
    \end{center}
    \item{Full Conditional for $\{\theta\}$}
%\subsubsection{$p(\{\theta\})|.)$}
    \begin{center}
    \begin{multline}
        \displaystyle
        log(p(\{\theta\}|.)) \propto \sum_{i=1}^n \bigg\{ \sum_{j=1}^T\Big\{
            (E^*_{ij}log(\pi_{ij}^{(SE)})) + (f_S(S^*, E^*, A_0)_{ij} - E^*_{ij})log(1-\pi_{ij}^{(SE)})\Big\}\bigg\} \\
    \end{multline}
    \end{center}

    \item{Full Conditional for $\{\beta\}$}
%\subsubsection{$p(\{\beta\}|.)$}
    \begin{center}
    \begin{multline}
        \displaystyle
        log(p(\{\beta\}|.)) \propto
            \sum_{i=1}^n \bigg\{ \sum_{j=1}^T\Big\{
            (E^*_{ij}log(\pi_{ij}^{(SE)})) + (f_S(S^*, E^*, A_0)_{ij} - E^*_{ij})log(1-\pi_{ij}^{(SE)})\Big\}\bigg\} \\
             + \sum_{k = 1}^K\bigg\{-\frac{\beta^2_k}{10}\bigg\}
    \end{multline}
    \end{center}
%\subsubsection{$p(\rho|.)$}

    \item{Full Conditional for $\rho$}
    \begin{center}
    \begin{multline}
        \displaystyle
        log(p(\rho|.)) \propto\sum_{i=1}^n \bigg\{ \sum_{j=1}^T\Big\{
            (E^*_{ij}log(\pi_{ij}^{(SE)})) + (f_S(S^*, E^*, A_0)_{ij} - E^*_{ij})log(1-\pi_{ij}^{(SE)})\Big\}\bigg\} \\
            + (log(\pi(\rho)))\\
    \end{multline}
    \end{center}

\item{Full Conditional for $\{\beta_{\pi_{j}^{(RS)}}\}$}
    \begin{center}
    \begin{multline}
        \displaystyle
        log(p(\{\pi_{j}^{(RS)}\}|.))\propto\sum_{j=1}^T\Big\{log(\pi_j^{(RS)})\Big[\sum_{i=1}^n\{S^*_{ij}\}\Big] \\
            \ \ \ \ + 
            log(1-\pi_j^{(RS)})\Big[\sum_{i=1}^n\{f_R(R^*, S^*, A_0)_{ij} - S^*_{ij}\}\Big]\Big\}\\ 
            + \bigg\{ -\frac{\tau^2_{RS}}{2}\|\beta_{\pi_j^{RS}}\|  \bigg\} \\ 
    \end{multline}
\end{center}

\item{Full Conditional for $\gamma^{(EI)}$} 
    \begin{center}
        \begin{multline}
            \displaystyle
            log(p(\gamma^{(EI)})) \propto log(\pi^{(EI)})\Big[\sum_{i=1}^n\sum_{j=1}^T\{I^*_{ij}\}\Big]  
            \\+ log(1-\pi^{(EI)})\Big[\sum_{i=1}^n\sum_{j=1}^T\{(f_E(E^*, I^*, A_0)_{ij} - I^*_{ij})\}\Big] + log(\pi(\gamma^{(IR)}))\\
        \end{multline}
    \end{center}
\item{Full Conditional for $\gamma^{(IR)}$} 
    \begin{center}
        \begin{multline}
            \displaystyle
            log(p(\gamma^{(IR)})) \propto log(\pi^{(IR)})\Big[\sum_{i=1}^n\sum_{j=1}^T\{R^*_{ij}\}\Big] \\ + 
            log(1-\pi^{(IR)})\Big[\sum_{i=1}^n\sum_{j=1}^T\{ f_I(I^*, R^*, A_0)_{ij} - R^*_{ij}\} \Big] + log(\pi(\gamma^{(IR)}))\\
        \end{multline}
    \end{center}




\end{itemize}

\section{MCMC Samplers: Current and Future Work}

While various combinations of slice and Metropolis-Hastings sampling have been explored, the current recommendation is to 
use blocked Metropolis-Hastings for all parameters. As implemented here, each of the full conditional distributions is sampled
by proposing a new value centered on the current value for the entire parameter. Coupled with using the auto-tuning functions to 
maintain a reasonable acceptance rate, this has different effects for compartments and other parameters. For compartments (because
they are integers), such an approach has the effect of choosing a random sample of time-locations to update, moving up or down by 
a small ammount (1 or 2). For other parameters, the entire parameter vector is either updated or not at each step.  

More development is planned to deal with the considerable autocorrelation issues experienced so far. 

See the tutorial documents for the most current implementation details. 

\section{OpenCL: Performance Tradeoffs}

OpenCL is a useful way to introduce parallelism to computationally intensive problems, and can target multi-core CPU's as well as
GPU's. Indeed, initial results suggest that for moderately large problems (20-30k time/location values) OpenCL can speed up the 
work of libspatialSEIR by around \%20. This number is expected to increase as more of the library is parallelized. 

On the other hand, for small problems OpenCL has enough overhead to significantly slow down run times. The Google Flu tutorial document 
gives some example code which is useful for checking whether parallelism is the best option for your particular problem. You can easily 
switch between OpenCL devices and single core CPU mode at any time during sampling. 

\section{R Level API Summary}

Once the R level API is complete, this section will include proper documentation and example code. For now, development is moving too fast
to keep a current summary. All example code and tutorial documents are kept up to date, so that's probably the best resource for now. 

\end{document}
