\documentclass[12pt]{article}
\oddsidemargin .15in \evensidemargin .0in \topmargin 0in
\columnsep 10pt \columnseprule 0pt
\marginparwidth 19pt \marginparsep 11pt \marginparpush 5pt
\headheight 0pt \headsep 0pt
\textheight 8in
\textwidth 6in 
\pagestyle{plain}
\usepackage{graphicx, latexsym}
\usepackage{alltt}
\usepackage{float}
\usepackage{amsmath}

\newcommand \noi {\noindent}
\newcommand \itx {\indent \indent \indent}
\newcommand \bg {\begin}
\newcommand \en {\end}
\newcommand \mth {\begin{math}}
\newcommand \mthx {\end{math}}
\newcommand \ds {\displaystyle}
\newcommand \mbreak {\\ \vspace{0.1in}}
\begin{document}

\begin{center}    
    \noi \bf SEIR Derivation and Algorithm\\
    \vspace{.05in}
    \noi Spring 2014\\
    \vspace{.05in}

    \vspace{.15in}
    \noi Grant Brown\\ 

\end{center}

\section{Problem Set Up}
\subsection{SEIR Structure}

    \subsubsection{Compartments}
        Denote the spatial locations (cities) $\left\{s_i : i = 1, ...,n \right\}$ \mbreak
        Let $d(s_i, s_l) = d_{il}$ define a measure of distance between 
        spatial locations $s_i$ and $s_l$. $d(s_i, s_i) = 0$, and $d(s_i, s_l) = d(s_l, s_i)$ \mbreak
        Let time (in weeks) be denoted ${t_j : j = 1, ...,T}$ \mbreak
        $\bf{N_j} = (N_{1j}, ..., N_{nj})$ is the vector of population sizes at time i \mbreak
        $\bf{S_j} = (S_{1j}, ..., S_{nj})$ is the vector of susceptable counts at time i \mbreak
        $\bf{E_j} = (E_{1j}, ..., E_{nj})$ is the vector of exposed counts at time i \mbreak
        $\bf{I_j} = (I_{1j}, ..., I_{nj})$ is the vector of infectious counts at time i \mbreak
        $\bf{R_j} = (R_{1j}, ..., R_{nj})$ is the vector of removed counts at time i. \mbreak
        $\bf{N_j} = \bf{S_j + E_j + I_j + R_j}$ for all $j$

    \subsubsection{Disease Evolution Process Model}
    \begin{center}
        $\bf{S_{j+1}} = \bf{S_j} - \bf{E_j^*} + \bf{S_j^*}$\mbreak
        $\bf{E_{j+1}} = \bf{E_j} - \bf{I_j^*} + \bf{E_j^*}$\mbreak
        $\bf{I_{j+1}} = \bf{I_j} - \bf{R_j^*} + \bf{I_j^*}$\mbreak
        $\bf{R_{j+1}} = \bf{R_j} - \bf{S_j^*} + \bf{R_j^*}$\mbreak
    \end{center}
    \vspace{0.15in}

    \vspace{0.15in}
        \noi
        $\{S_{ij}^*\} \sim\ iid\  binom(R_{ij}, \pi_j^{(RS)})$\\
        $\{E_{ij}^*\} \sim binom(S_{ij}, \pi^{SE}_{ij})\\$
        $\{I_{ij}^*\} \sim\ iid\  binom(E_{ij}, \pi^{(EI)})$\\
        $\{R_{ij}^*\} \sim\ iid\  binom(I_{ij}, \pi^{(IR)})$\\



$\pi^{EI}$, and $\pi^{IR}$ are easily modeled with standard prior distributions. 
More care must be given to the development of a model for the $\left\{\pi^{SE}_{ij} \right\}$ and $\left\{\pi^{RS}_j \right\}$, the first of
which describes the actual infection process and must account for predictor variables as well as the 
spatial structure of $\left\{ s_i \right\}$, and the second of which captures the diminishing immunity to particular strains of illness. 

\subsection{Infection Process - CAR Model Motivation}

Consider the process by which people become infected with a communicable disease. 
Namely, consider the situation in which a person `A' has contacted another person, `B', 
who is infectious (for some suitable definition of contacted). 
Let $p$ be the probability that person `A' becomes infected with the disease, and
let $q=1-p$. Now we introduce a number of assumptions:

\begin{itemize}

    \item Assume that the number of `contacts' $K_i$ between a person of interest 
    and other individuals within a spatial unit $s_i$ at a given time point follows a poisson 
    distribution:\\
    \begin{center}
        $K_j \sim Po(\lambda_i)$
    \end{center}
    \item Assume that when individuals travel to other spatial locations, their 
        contact behavior is well modeled by the contact behavior of that spatial unit (when in Rome).  
    \item Contact between spatial locations is proportional to some known function $f(d_{il})$
        of the driving distance (in minutes) between the centroids of $s_i$ and $s_l$
    \item Unknown and unmeasureable mixing with external populations results in some overall
        probability of infection for each person in a population. Parameterize this probability as $1-exp(-\gamma_j)$, and let 
        the $Inf(ext,t_j)$ denote the infection event such that $P(Inf(ext,t_j)) = 1-exp(-\gamma_j)$ and 
        $P(!Inf(ext,t_j)) = exp(-\gamma_j)$.
\end{itemize}


Define $\delta_{ij}$ to be the proportion of persons who are infectious in spatial unit $s_i$ 
at time $t_j$. Then, letting $Inf(s_i, t_j)$ be the event that a person becomes 
infected from contact within spatial unit $s_i$ at time $t_j$, we can derive:

\begin{center}

    $P(Inf(.,t_j)) = 1 - P(!Inf(s_i, t_j)) \cdot  P(!Inf(s_{-i}, t_j)) \cdot P(!Inf(ext))$ \mbreak
    where \mbreak 
    $P(!Inf(s_i, t_j)) = E(!Inf(s_i, t_j)) = E(E(!Inf(s_i, t_j)|K_i=k_i))$  \mbreak 
    $\displaystyle =E(((1-\delta_{ij})q)^{k_i})$  \mbreak 
    $\displaystyle = \sum_{k=0}^{\infty} ((1-\delta_{ij})q)^k(\frac{\lambda_i^ke^{-\lambda_i}}{k!})$\mbreak
    $\displaystyle =  \sum_{k=0}^{\infty} q_{ij}^k (\frac{\lambda_i^ke^{-\lambda_i}}{k!})$\mbreak
    $\displaystyle = \frac{e^{-\lambda_i}}{e^{-q_{ij}\lambda_i}}(1)$
    $ = e^{-\lambda_i\cdot(1-q_{ij})} $
    $ = e^{-\lambda_i \cdot p_{ij}} $
    $ = e^{-\lambda_i \cdot (\delta_{ij}p)}$ \mbreak
    Therefore,  $P(Inf(s_i, t_j)) = 1 - e^{-\lambda_i \cdot (\delta_{ij}p)} $ \mbreak
    Similarly,\mbreak
    $\displaystyle P(!Inf(s_{-i}, t_j)) = \prod_{\left\{l \ne i\right\}}\left[P(!Inf(s_l, t_j)) \right]$ \mbreak
    $\displaystyle = \prod_{\left\{ l \ne i \right\}}\left[E(!Inf(s_{-i}, t_j))\right] $
    $\displaystyle = \prod_{\left\{ l \ne i \right\}}\left[E(E(!Inf(s_{-i}, t_j)|K_i=k_i)) \right]$  \mbreak 
    $\displaystyle = \prod_{\left\{ l \ne i \right\}}\left[ E((1-\delta_{lj})q)^k )  \right]$\mbreak 
    $\displaystyle = \prod_{\left\{ l \ne i \right\}}\left[ \sum_{k=0}^{\infty}(q_{lj}(i))^k\frac{(\lambda_l\cdot f(d_{il}))^ke^{-\lambda_l \cdot f(d_{il})}}{k!}    \right] $
    $\displaystyle = \prod_{\left\{ l \ne i \right\}}\left[ \frac{e^{-\lambda_l \cdot f(d_{il})}}{e^{-q_{lj}\lambda_l f(d_{il})}}(1)  \right]$\mbreak
    $\displaystyle = \prod_{\left\{ l \ne i \right\}}\left[ e^{-\lambda_l \cdot f(d_{il}) p_{lj}}  \right]$
    $\displaystyle = \prod_{\left\{ l \ne i \right\}}\left[ e^{-\lambda_l\cdot f(d_{il}) \cdot (\delta_{lj}p) }  \right]$\mbreak
    $\displaystyle = exp\left\{\sum_{\left\{ l \ne i \right\}}\left[p\lambda_l\delta_{jl}f(d_{il})   \right] \right\}$ \mbreak
    Thus, for the probabilty of infection for a person living in $s_i$ at time $t_j$ we have: \mbreak 
    $\displaystyle 1-\left(e^{-\lambda_i \cdot (\delta_{ij}p)}\right) \left( 
        e^{\left\{\sum_{\left\{ l \ne i \right\}}\left[p\lambda_l\delta_{jl}f(d_{il})   \right] \right\}}\right)\left(e^{-\gamma_j}\right)
     $ \mbreak 

    $\displaystyle = 1- exp\left\{-\gamma_j-\delta_{ij}e^{\theta_{i}} - \sum_{\left\{ l \ne i  \right\}}
        (f(d_{il})\delta_{il}e^{\theta_{l}})   \right\}$ 
    , where $\theta_{v} = log(\lambda_{v}p)$
 
\end{center}

Notes: 
\begin{enumerate}
    \item By defining $f(d_{ii})$ to be equal to $1$ for all $i$, the above expression has a simple 
        matrix form. 
    \item WLOG, we can make the mixing parameters dependent on space and time, defining $\lambda_{v_1, v_2}$,
        and correspondingly $\theta_{v_1,v_2}$

\end{enumerate}
To model the $\pi_j^{RS}$, some structure of lower than $T$ dimensions is desireable to reduce potential 
identifiability issues. A covariate structure constructed to capture natural variation in this quantity 
(seasonal polynomial components for example) will do nicely. Let $X(\pi_j^{RS})$ and $\beta_{\pi^{RS}}$ 
denote the covariate vector and corresponding regression parameter estimates 
for the j'th R to S transition probability respectively. Any number of link functions would be appropriate here.
 
\begin{enumerate}
    \item (Complimentary) Log Link \\
    $log(\pi_j^{RS}) = -X(\pi_j^{RS}) \beta_{\pi^{RS}}$
    \item Complimentary Log-Log Link \\
    $log(-log(\pi_j^{RS})) = X(\pi_j^{RS}) \beta_{\pi^{RS}}$
    \item Logit Link \\
    $log(\frac{\pi_j^{RS}}{1-\pi_j^{RS}}) = X(\pi_j^{RS}) \beta_{\pi^{RS}}$
\end{enumerate}

For simplicity, and to parallel the covariate scale used for $\left\{\pi_{ij}^{SE} \right\}$, lets 
use the log link (or rather, complimentary log link to keep the parameter sign easily interpretable).

\section{Posterior Distribution and Full Conditionals}

\subsection{Posterior Distribution}
\subsubsection{Summary of Distribution Components}
$\{S_{ij}^*\} \sim\  binom(R_{ij}, \pi_j^{(RS)})$\mbreak
$\{E_{ij}^*\} \sim binom(S_{ij}, \pi^{SE}_{ij})$ \mbreak
$\{I_{ij}^*\} \sim\  binom(E_{ij}, \pi^{(EI)})$\mbreak
$\{R_{ij}^*\} \sim\  binom(I_{ij}, \pi^{(IR)})$\mbreak
\noi
$\left\{\pi_j^{RS}  \right\} \sim (iid)\ beta(0.5, 0.5)$\mbreak 
$\left\{\pi^{EI}  \right\} \sim beta(0.5, 0.5)$\mbreak 
%%$\left\{\pi_j^{IR}  \right\} \sim beta(0.5, 0.5)$\mbreak 
$\left\{ \theta_{ij}\right\} \sim \mathcal{N}(\eta_{ij}, \sigma^2_{\theta})$ \mbreak
$\left\{ \beta \right\} \sim \mathcal{N}(0, \tau^2_\beta) $\mbreak 
$\left\{ \beta_{\pi^{RS}} \right\} \sim \mathcal{N}(0, \tau^2_{RS}) $\mbreak 
$\sigma^2_{\theta} \sim \Gamma(\alpha_\theta, \beta_\theta)$\mbreak
$\gamma_j \sim \Gamma(\alpha_\gamma, \beta_\gamma)$\mbreak
$\rho \sim U(0,1)$

\subsubsection{Deterministic Functions}
$S = f_S(S_0, E^*_0, S^*_0, S^*, E^*)$ \mbreak
$E = f_E(E_0, I^*_0, E^*_0, E^*, I^*)$ \mbreak
$I = f_I(I_0, R^*_0, I^*_0, I^*, R^*)$ \mbreak
$R = f_R(R_0, S^*_0, R^*_0, R^*, S^*)$ \mbreak
$\displaystyle log(\pi^{(SE)}_{ij}) = -\gamma_j-\delta_{ij}e^{\theta_{ij}} - \sum_{\left\{ l \ne i \right\}}d_{il}\delta_{il}e^{\theta_{il}}$\mbreak
$log(\pi_j^{RS}) = -X(\pi_j^{RS}) \beta_{\pi^{RS}}$\mbreak
$d_{il} = f(\rho, s_i, s_l)$\mbreak
$\delta_{ij} = \frac{I_{ij}}{N_{ij}}$ \mbreak
$\eta_{ij} = X_{ij}\beta$
\subsubsection{Posterior Distribution}

\begin{center}
\begin{multline}
\displaystyle
log(p(\theta,\beta,\rho,S^*,E^*,R^*|.)) \propto \Bigg[ 
    \sum_{i=1}^n \bigg\{ \sum_{j=1}^T
        \Big\{
            (S^*_{ij}log(\pi_j^{(RS)}) + (R_{ij} - S^*_{ij})log(1-\pi_j^{(RS)})) \\
            + (E^*_{ij}log(\pi_{ij}^{(SE)})) + (S_{ij} - E^*_{ij})log(1-\pi_{ij}^{(SE)}) \\
            + (I^*_{ij}log(\pi^{(EI)})) + (E_{ij} - I^*_{ij})log(1-\pi^{(EI)}) \\
            + (R^*_{ij}log(\pi^{(IR)})) + (I_{ij} - R^*_{ij})log(1-\pi^{(IR)}) \\
    - \frac{1}{2}log(\sigma^2_{\theta}) - \frac{(\theta_{ij}-\eta_{ij})^2}{2\sigma^2_{\theta}}\Big\}\bigg\} \\
    + \sum_{k = 1}^K\bigg\{-\frac{\beta^2_k}{10}\bigg\}
            + (log(\pi(\sigma^2_{\theta})))
            + (log(\pi(\rho)))\\
            + \bigg\{ -\frac{\tau^2_{RS}}{2}\|\beta_{\pi_j^{RS}}\|  \bigg\} \\ 
            + (\frac{1}{2}log(\pi^{(EI)}) + \frac{1}{2}log(1-\pi^{(EI)})) \\
            + (\frac{1}{2}log(\pi^{(IR)}) + \frac{1}{2}log(1-\pi^{(IR)}))    \\
            + \sum_{j=1}^{T} \bigg\{ log(\pi(\gamma_j)) \bigg\} \Bigg]\\
\end{multline}
\end{center}

For simplicity, and because the model is sufficiently flexible without the added over-dispersion, set $\theta_{ij} = \eta_{ij}$ with probability 1. This 
gives the following simplified posterior distribution:

\begin{center}
\begin{multline}
\displaystyle
log(p(\theta,\beta,\rho,S^*,E^*,R^*|.)) \propto \Bigg[ 
    \sum_{i=1}^n \bigg\{ \sum_{j=1}^T
        \Big\{
            (S^*_{ij}log(\pi_j^{(RS)}) + (R_{ij} - S^*_{ij})log(1-\pi_j^{(RS)})) \\
            + (E^*_{ij}log(\pi_{ij}^{(SE)})) + (S_{ij} - E^*_{ij})log(1-\pi_{ij}^{(SE)}) \\
            + (I^*_{ij}log(\pi^{(EI)})) + (E_{ij} - I^*_{ij})log(1-\pi^{(EI)}) \\
            + (R^*_{ij}log(\pi^{(IR)})) + (I_{ij} - R^*_{ij})log(1-\pi^{(IR)}) \Big\}\bigg\} \\
    + \sum_{k = 1}^K\bigg\{-\frac{\beta^2_k}{10}\bigg\}
            + (log(\pi(\rho)))\\
            +  \bigg\{ -\frac{\tau^2_{RS}}{2}\|\beta_{\pi_j^{RS}}\|  \bigg\} \\ 
            + (\frac{1}{2}log(\pi^{(EI)} + \frac{1}{2}log(1-\pi^{(EI)}))) \\
            + (\frac{1}{2}log(\pi^{(IR)} + \frac{1}{2}log(1-\pi^{(IR)}))) \\
            + \sum_{j=1}^{T} \bigg\{ log(\pi(\gamma_j)) \bigg\}\Bigg]\\
\end{multline}
\end{center}

Now define the fixed set $A_0 = \{S_0, E_0, I_0, R_0, S^*_0, E^*_0, I^*_0, R^*_0\}$,
and reparameterize $S, E, I, R$ as explicit functions of the transition matrices. 


\begin{center}
\begin{multline}
\displaystyle
log(p(\theta,\beta,\rho,S^*,E^*,R^*|.)) \propto\\ \Bigg[ 
            %% S Piece
            \sum_{j=1}^T\Big\{log(\pi_j^{(RS)})\sum_{i=1}^n\{S^*_{ij}\} + 
            log(1-\pi_j^{(RS)})\sum_{i=1}^n\{f_R(R^*, S^*, A_0)_{ij} - S^*_{ij}\}\Big\} \\ 
            %% E Piece
            +\sum_{i=1}^n \bigg\{ \sum_{j=1}^T\Big\{
            (E^*_{ij}log(\pi_{ij}^{(SE)})) + (f_S(S^*, E^*, A_0)_{ij} - E^*_{ij})log(1-\pi_{ij}^{(SE)})\Big\}\bigg\} \\
            %% I Piece
            + log(\pi^{(EI)})\sum_{i=1}^n\sum_{j=1}^T\{I^*_{ij}\} 
            + log(1-\pi^{(EI)})\sum_{i=1}^n\sum_{j=1}^T\{(f_E(E^*, I^*, A_0)_{ij} - I^*_{ij})\}\\
            %% R Piece
            + log(\pi^{(IR)})\sum_{i=1}^n\sum_{j=1}^T\{R^*_{ij}\} + 
            log(1-\pi^{IR})\sum_{i=1}^n\sum_{j=1}^T\{ f_I(I^*, R^*, A_0)_{ij} - R^*_{ij}\}\\
            %% Beta
            + \sum_{k = 1}^K\bigg\{-\frac{\beta^2_k}{10}\bigg\}
            %% Rho prior
            + (log(\pi(\rho)))\\
            %% \beta_{\pi_j^{RS}}
            + \bigg\{ -\frac{\tau^2_{RS}}{2}\|\beta_{\pi_j^{RS}}\|  \bigg\} \\ 
            %% \pi^{EI}
            + (\frac{1}{2}log(\pi^{(EI)} + \frac{1}{2}log(1-\pi^{(EI)}))) \\
            %% \pi^{IR}
            + (\frac{1}{2}log(\pi^{(IR)} + \frac{1}{2}log(1-\pi^{(IR)})))   
            %% Gamma
            + \sum_{j=1}^{T} \bigg\{ log(\pi(\gamma_j)) \bigg\}\Bigg]\\
\end{multline}
\begin{multline}
\displaystyle
= \Bigg[ 
            %% S Piece
    \sum_{j=1}^T\Big\{log(\pi_j^{(RS)})\Big[\sum_{i=1}^n\{S^*_{ij}\}\Big] + 
            log(1-\pi_j^{(RS)})\Big[\sum_{i=1}^n\{f_R(R^*, S^*, A_0)_{ij} - S^*_{ij}\}\Big]\Big\}\\ 
            %% E Piece
            +\sum_{i=1}^n \bigg\{ \sum_{j=1}^T\Big\{
            (E^*_{ij}log(\pi_{ij}^{(SE)})) + (f_S(S^*, E^*, A_0)_{ij} - E^*_{ij})log(1-\pi_{ij}^{(SE)})\Big\}\bigg\} \\
            %% I Piece
            + log(\pi^{(EI)})\Big[\frac{1}{2} + \sum_{i=1}^n\sum_{j=1}^T\{I^*_{ij}\}\Big] 
            + log(1-\pi^{(EI)})\Big[\frac{1}{2} + \sum_{i=1}^n\sum_{j=1}^T\{(f_E(E^*, I^*, A_0)_{ij} - I^*_{ij})\}\Big]\\
            %% R Piece
            + log(\pi^{(IR)})\Big[\frac{1}{2} + \sum_{i=1}^n\sum_{j=1}^T\{R^*_{ij}\}\Big] + 
            log(1-\pi^{(IR)})\Big[\frac{1}{2} + \sum_{i=1}^n\sum_{j=1}^T\{ f_I(I^*, R^*, A_0)_{ij} - R^*_{ij}\}\Big]\\
            %% Beta
            + \sum_{k = 1}^K\bigg\{-\frac{\beta^2_k}{10}\bigg\}
            %% Rho prior
            + (log(\pi(\rho)))\\
            %% \beta_{\pi_j^{RS}}
            + \bigg\{ -\frac{\tau^2_{RS}}{2}\|\beta_{\pi_j^{RS}}\|  \bigg\} \\ 
            %% Gamma prior
            + \sum_{j=1}^{T} \bigg\{ log(\pi(\gamma_j)) \bigg\}\\
\end{multline}

\end{center}


\subsection{Full Conditional Distributions}
Depending on when the disease of interest is expected to be detected, the data can be 
introduced as the $E^*$ exposure counts, or as the $I^*$ infection counts. Other latent 
processes could also be introduced. For this reason, full conditional distributions are 
introduced for all categories. 

\begin{itemize}
    \item{Full Conditional For $S^*$}
%\subsubsection{$p(S^*|.)$}
    \begin{center}
        \begin{multline}
        \displaystyle
        log(p(S^*|.))\propto 
            \sum_{j=1}^T\Big\{log(\pi_j^{(RS)})\sum_{i=1}^n\{S^*_{ij}\} + 
                log(1-\pi_j^{(RS)})\sum_{i=1}^n\{f_R(R^*, S^*, A_0)_{ij} - S^*_{ij}\}\Big\}\\ 
                +\sum_{i=1}^n \bigg\{ \sum_{j=1}^T\Big\{
                f_S(S^*, E^*, A_0)_{ij}log(1-\pi_{ij}^{(SE)})\Big\}\bigg\}
        \end{multline}
    \end{center}

    \item{Full Conditional for $E^*$}
%\subsubsection{$p(E^*|.)$}
    \begin{center}
    \begin{multline}
        \displaystyle
        log(p(E^*|.))\propto \sum_{i=1}^n \bigg\{ \sum_{j=1}^T\Big\{
                (E^*_{ij}log(\pi_{ij}^{(SE)})) + (f_S(S^*, E^*, A_0)_{ij} - E^*_{ij})log(1-\pi_{ij}^{(SE)})\Big\}\bigg\} \\
                + log(1-\pi^{(EI)})\sum_{i=1}^n\sum_{j=1}^T\{(f_E(E^*, I^*, A_0)_{ij}\}\\
    \end{multline}
    \end{center}

    \item{Full Conditional for $I^*$}
%\subsubsection{$p(I^*|.)$}
    \begin{center}
    \begin{multline}
        \displaystyle
        log(p(I^*|.)) \propto log(\pi^{(EI)})\sum_{i=1}^n\sum_{j=1}^T\{I^*_{ij}\} 
            + log(1-\pi^{(EI)})\sum_{i=1}^n\sum_{j=1}^T\{(f_E(E^*, I^*, A_0)_{ij} - I^*_{ij})\}\\
            + log(1-\pi^{IR})\sum_{i=1}^n\sum_{j=1}^T\{ f_I(I^*, R^*, A_0)_{ij}\}\\
    \end{multline}
    \end{center}

    \item{Full Conditional for $R^*$}
%\subsubsection{$p(R^*|.)$}
    \begin{center}
    \begin{multline}
        \displaystyle
        log(p(R^*|.)) \propto log(\pi^{(IR)})\sum_{i=1}^n\sum_{j=1}^T\{R^*_{ij}\} + 
            log(1-\pi^{IR})\sum_{i=1}^n\sum_{j=1}^T\{ f_I(I^*, R^*, A_0)_{ij} - R^*_{ij}\}\\
            +   \sum_{j=1}^T\Big\{log(1-\pi_j^{(RS)})\sum_{i=1}^n\{f_R(R^*, S^*, A_0)_{ij}\}\Big\}\\ 
    \end{multline}
    \end{center}
    \item{Full Conditional for $\{\theta\}$}
%\subsubsection{$p(\{\theta\})|.)$}
    \begin{center}
    \begin{multline}
        \displaystyle
        log(p(\{\theta\}|.)) \propto \sum_{i=1}^n \bigg\{ \sum_{j=1}^T\Big\{
            (E^*_{ij}log(\pi_{ij}^{(SE)})) + (f_S(S^*, E^*, A_0)_{ij} - E^*_{ij})log(1-\pi_{ij}^{(SE)})\Big\}\bigg\} \\
    \end{multline}
    \end{center}

    \item{Full Conditional for $\{\beta\}$}
%\subsubsection{$p(\{\beta\}|.)$}
    \begin{center}
    \begin{multline}
        \displaystyle
        log(p(\{\beta\}|.)) \propto
            \sum_{i=1}^n \bigg\{ \sum_{j=1}^T\Big\{
            (E^*_{ij}log(\pi_{ij}^{(SE)})) + (f_S(S^*, E^*, A_0)_{ij} - E^*_{ij})log(1-\pi_{ij}^{(SE)})\Big\}\bigg\} \\
             + \sum_{k = 1}^K\bigg\{-\frac{\beta^2_k}{10}\bigg\}
    \end{multline}
    \end{center}
%\subsubsection{$p(\rho|.)$}

    \item{Full Conditional for $\rho$}
    \begin{center}
    \begin{multline}
        \displaystyle
        log(p(\rho|.)) \propto\sum_{i=1}^n \bigg\{ \sum_{j=1}^T\Big\{
            (E^*_{ij}log(\pi_{ij}^{(SE)})) + (f_S(S^*, E^*, A_0)_{ij} - E^*_{ij})log(1-\pi_{ij}^{(SE)})\Big\}\bigg\} \\
            + (log(\pi(\rho)))\\
    \end{multline}
    \end{center}

    \item{Full Conditional for $\gamma_j$}
    \begin{center}
    \begin{multline}
        \displaystyle
        log(p(\rho|.)) \propto\sum_{i=1}^n \bigg\{ \sum_{j=1}^T\Big\{
            (E^*_{ij}log(\pi_{ij}^{(SE)})) + (f_S(S^*, E^*, A_0)_{ij} - E^*_{ij})log(1-\pi_{ij}^{(SE)})\Big\}\bigg\} \\
            + (\sum_{j=1}^{T}log(\pi(\gamma_j)))\\
    \end{multline}
    \end{center}


\item{Full Conditional for $\{\beta_{\pi_{j}^{(RS)}}\}$}
    \begin{center}
    \begin{multline}
        \displaystyle
        log(p(\{\pi_{j}^{(RS)}\}|.))\propto\sum_{j=1}^T\Big\{log(\pi_j^{(RS)})\Big[\sum_{i=1}^n\{S^*_{ij}\}\Big] \\
            \ \ \ \ + 
            log(1-\pi_j^{(RS)})\Big[\sum_{i=1}^n\{f_R(R^*, S^*, A_0)_{ij} - S^*_{ij}\}\Big]\Big\}\\ 
            + \bigg\{ -\frac{\tau^2_{RS}}{2}\|\beta_{\pi_j^{RS}}\|  \bigg\} \\ 
    \end{multline}
\end{center}

\item{Full Conditional for $\pi^{(EI)}$} 
    \begin{center}
        \begin{multline}
            \displaystyle
            log(p(\pi^{(EI)})) \propto log(\pi^{(EI)})\Big[\frac{1}{2} + \sum_{i=1}^n\sum_{j=1}^T\{I^*_{ij}\}\Big] 
            \\+ log(1-\pi^{(EI)})\Big[\frac{1}{2} + \sum_{i=1}^n\sum_{j=1}^T\{(f_E(E^*, I^*, A_0)_{ij} - I^*_{ij})\}\Big]\\
            \Rightarrow \pi^{(EI)} \sim beta(\frac{3}{2} + \sum_{j=1}^T\sum_{i=1}^n\{I^*_{ij}\}, 
            \frac{3}{2} + \sum_{j=1}^T\sum_{i=1}^n\{f_E(E^*, I^*, A_0)_{ij}\} - \sum_{j=1}^T\sum_{i=1}^n\{I^*_{ij}\}) 
        \end{multline}
    \end{center}
\item{Full Conditional for $\pi^{(IR)}$} 
    \begin{center}
        \begin{multline}
            \displaystyle
            log(p(\pi^{(IR)})) \propto log(\pi^{(IR)})\Big[\frac{1}{2} + \sum_{i=1}^n\sum_{j=1}^T\{R^*_{ij}\}\Big] + 
            log(1-\pi^{(IR)})\Big[\frac{1}{2} + \sum_{i=1}^n\sum_{j=1}^T\{ f_I(I^*, R^*, A_0)_{ij} - R^*_{ij}\}\Big]\\
            \Rightarrow \pi^{(IR)} \sim beta(\frac{3}{2} + \sum_{j=1}^T\sum_{i=1}^n\{R^*_{ij}\}, 
            \frac{3}{2} + \sum_{j=1}^T\sum_{i=1}^n\{f_I(I^*, R^*, A_0)_{ij}\} - \sum_{j=1}^T\sum_{i=1}^n\{R^*_{ij}\})
        \end{multline}
    \end{center}




\end{itemize}


\section{Notes}

\subsection{Random Variate Generation}

The gamma distribution is parameterized by boost as:
$\displaystyle p(x) = x^{\alpha-1}\frac{e^{-x/\beta}}{\beta^\alpha\Gamma(\alpha)}$

\end{document}
